\documentclass[11pt, a4paper]{article}
%\usepackage{geometry}
\usepackage[inner=1.5cm,outer=1.5cm,top=2.5cm,bottom=2.5cm]{geometry}
\pagestyle{empty}
\usepackage{graphicx}
\usepackage{fancyhdr, lastpage, bbding, pmboxdraw}
\usepackage[usenames,dvipsnames]{color}
\definecolor{darkblue}{rgb}{0,0,.6}
\definecolor{darkred}{rgb}{.7,0,0}
\definecolor{darkgreen}{rgb}{0,.6,0}
\definecolor{red}{rgb}{.98,0,0}
\usepackage[colorlinks,pagebackref,pdfusetitle,urlcolor=darkblue,citecolor=darkblue,linkcolor=darkred,bookmarksnumbered,plainpages=false]{hyperref}
\renewcommand{\thefootnote}{\fnsymbol{footnote}}

\pagestyle{fancyplain}
\fancyhf{}
\lhead{ \fancyplain{}{Machine Learning} }
%\chead{ \fancyplain{}{} }
\rhead{ \fancyplain{}{Spring, 2017} }
%\rfoot{\fancyplain{}{page \thepage\ of \pagafaeref{LastPage}}}
\fancyfoot[RO, LE] {page \thepage\ of \pageref{LastPage} }
\thispagestyle{plain}

%%%%%%%%%%%% LISTING %%%
\usepackage{listings}
\usepackage{caption}
\DeclareCaptionFont{white}{\color{white}}
\DeclareCaptionFormat{listing}{\colorbox{gray}{\parbox{\textwidth}{#1#2#3}}}
\captionsetup[lstlisting]{format=listing,labelfont=white,textfont=white}
\usepackage{verbatim} % used to display code
\usepackage{fancyvrb}
\usepackage{acronym}
\usepackage{amsthm}
\VerbatimFootnotes % Required, otherwise verbatim does not work in footnotes!



\definecolor{OliveGreen}{cmyk}{0.64,0,0.95,0.40}
\definecolor{CadetBlue}{cmyk}{0.62,0.57,0.23,0}
\definecolor{lightlightgray}{gray}{0.93}



\lstset{
%language=bash,                          % Code langugage
basicstyle=\ttfamily,                   % Code font, Examples: \footnotesize, \ttfamily
keywordstyle=\color{OliveGreen},        % Keywords font ('*' = uppercase)
commentstyle=\color{gray},              % Comments font
numbers=left,                           % Line nums position
numberstyle=\tiny,                      % Line-numbers fonts
stepnumber=1,                           % Step between two line-numbers
numbersep=5pt,                          % How far are line-numbers from code
backgroundcolor=\color{lightlightgray}, % Choose background color
frame=none,                             % A frame around the code
tabsize=2,                              % Default tab size
captionpos=t,                           % Caption-position = bottom
breaklines=true,                        % Automatic line breaking?
breakatwhitespace=false,                % Automatic breaks only at whitespace?
showspaces=false,                       % Dont make spaces visible
showtabs=false,                         % Dont make tabls visible
columns=flexible,                       % Column format
morekeywords={__global__, __device__},  % CUDA specific keywords
}

%%%%%%%%%%%%%%%%%%%%%%%%%%%%%%%%%%%%
\begin{document}
\begin{center}
{\Large \textsc{Machine Learning}}
\end{center}
\begin{center}
Spring 2017
\end{center}
%\date{September 26, 2014}

\begin{center}
\rule{6in}{0.4pt}
\begin{minipage}[t]{.75\textwidth}
\begin{tabular}{llcccll}
\textbf{Contact:} & Patrick Beukema & & &  & \textbf{Time:} & W 12:00 -- 1:00 \\
\textbf{Email:} &  \href{plb23@pitt.edu}{plb23@pitt.edu} & & & & \textbf{Place:} & Mellon Institute 115
\end{tabular}
\end{minipage}
\rule{6in}{0.4pt}
\end{center}
\vspace{.5cm}
\setlength{\unitlength}{1in}
\renewcommand{\arraystretch}{2}

\noindent\textbf{Course Page:}
\url{http://github.com/pbeukema/ML_CNBC}



\vskip.15in


\vskip.15in
\noindent\textbf{Main Text:} %\footnotemark
Each Week we will read and do exercises from one chapter of Machine Learning: A Probabilistic Perpspective. \href{https://www.cs.ubc.ca/~murphyk/MLbook/}{https://www.cs.ubc.ca/~murphyk/MLbook/}
\vskip.15in
\noindent\textbf{Schedule \& Exercises:}
\begin{enumerate}
\item Introduction 
\begin{itemize}
\item Excersies: 1.1-1.3, Rec: Write KNN from scratch
\end{itemize}
\item Probability
\begin{itemize}
\item Excersies: 2.1-2.5, 2.12, Rec: 2.17 (Prove \& Run simulation)
\end{itemize}
\item Generative models for discrete data
\begin{itemize}
\item Excersies
\end{itemize}
\item Gaussian models
\begin{itemize}
\item Excersies
\end{itemize}
\item Bayesian statistics
\begin{itemize}
\item Excersies
\end{itemize}
\item Frequestist statistics
\begin{itemize}
\item Excersies
\end{itemize}
\item Linear Regression
\begin{itemize}
\item Excersies
\end{itemize}
\item Logistic Regression
\begin{itemize}
\item Excersies
\end{itemize}
\item Generalized linaer models and the exponential family
\begin{itemize}
\item Excersies
\end{itemize}
\item Directed graphical models
\begin{itemize}
\item Excersies
\end{itemize}
\item Mixture models and the EM algorithm
\begin{itemize}
\item Excersies
\end{itemize}
\item Latent linear models
\begin{itemize}
\item Excersies
\end{itemize}
\item Sparse linear models
\begin{itemize}
\item Excersies
\end{itemize}

\item Kernels
\begin{itemize}
\item Excersies
\end{itemize}
\item Gaussian processes
\begin{itemize}
\item Excersies
\end{itemize}
\end{enumerate} 

% \footnotetext{Downloadable ebook versions are available on AeLP.}

\vskip.15in
\noindent\textbf{Course objectives:}  This course is  primarily designed for graduate students, and will introduce an audience to the state-of-the-art in modeling techniques for computer science and engineering majors. We try to discuss as many models as possible. We chiefly focus on complex networks, inference, machine learning, and probabilistic/statistical models and methods.



\vskip.15in
\noindent\textbf{Prerequisites:}
It will be challenging to follow along without some bacground in calculus, linear algebra, probability and statistics. 


\vspace*{.15in}


\vskip.15in
\noindent\textbf{Important Dates:}
\begin{center} \begin{minipage}{3.8in}
\begin{flushleft}
Midterm \#1      \dotfill ~\={A}b\={a}n 16, 1393 $\equiv$ November 7, 2014 \\
Midterm \#2      \dotfill ~\={A}zar 21, 1393 $\equiv$ December 12, 2014 \\
%Project Deadline \dotfill ~Month Day \\
Final Exam       \dotfill ~Dey 18, 1393 $\equiv$ January 8, 2015 \\
\end{flushleft}
\end{minipage}
\end{center}

\vskip.15in


\vskip.15in
\noindent\textbf{Resources \& Recommendations:}  
\begin{itemize}
\item Learning python:  Learn Python the Hard Way \href{https://learnpythonthehardway.org}{learnpythonthehardway.org/}
\item Programming: Jupyter notebooks \href{http://jupyter.org/}{jupyter.org}
\item Text editor (\& Programming): Atom \href{http://atom.io/ }{atom.io} 
\item Github: 
\end{itemize}

\vskip.15in



%%%%%% END 
\end{document} 